\documentclass[a4paper,twoside,12pt]{article}
\usepackage{amsmath}
\usepackage{amsfonts}
\usepackage{amssymb}
\usepackage[utf8]{inputenc}
\usepackage[T1]{fontenc}
\usepackage{fancyhdr}
\usepackage{stmaryrd}
\usepackage{textcomp}
\usepackage[french]{babel}
\usepackage{variations}
\usepackage{psfrag,graphicx}
\usepackage{array}
\usepackage{verbatim}
\usepackage{psfrag}
\usepackage{bbm}
\usepackage{ifthen}
\usepackage[hmarginratio=3:4]{geometry}


\newlength{\myhoffset}
\newlength{\mytextwidth}
\newlength{\myvoffset}
\newlength{\mytextheight}
\setlength{\myhoffset}{-0.75cm}
\setlength{\mytextwidth}{1.5cm}
\setlength{\myvoffset}{-1.5cm}
\setlength{\mytextheight}{3cm}

\addtolength{\hoffset}{\myhoffset}
\addtolength{\textwidth}{\mytextwidth} 
\addtolength{\voffset}{\myvoffset}
\addtolength{\textheight}{\mytextheight} 

%\setlength{\oddsidemargin}{5mm}
%\setlength{\evensidemargin}{5mm}

%textpos : permet de placer des éléments par leur coordonnées absolues sur la page
% utilisé pour les entêtes et pied de pages

\usepackage[absolute]{textpos}
\setlength{\TPHorizModule}{1mm}
\setlength{\TPVertModule}{\TPHorizModule}
\textblockorigin{-\myhoffset}{-\myvoffset}

% appendix : permet d'écrire en couleur

\usepackage[usenames,dvipsnames,table]{xcolor}

% fancyhdr : permet de personnaliser les entêtes et pieds de page des documents

\usepackage{fancyhdr}

%% pdfpages : utilisé pour insérer la page de garde avant le document

\usepackage{pdfpages}

% caption : centre les légendes des figures

\usepackage[center]{caption}

% wrapfig : permet de créer des figures entourées de texte

\usepackage{wrapfig}

%placeins : permet d'obliger Latex à insérer les figures en attente avant de continuer

\usepackage{placeins}

% subcaption : permet de créer des figures avec plusieurs sous-figures

\usepackage{subcaption}

% appendix : permet de gérer les annexes

\usepackage[page]{appendix}

% multibib : permet d'organiser la bibliographie par catégories

\usepackage{multibib}

% layouts : affichage d'une longueur

\usepackage{layouts}

% tabularx : pour le tableau d'Antoine
\usepackage{color}
\usepackage{colortbl,arydshln}
\usepackage{tabularx}

% auto-pst-pdf : nécessaire pour insérer des formules latex dans un .eps, commenter la ligne suivante si vous n'avez pas ce package

\usepackage{auto-pst-pdf}
\usepackage{cutwin}
\usepackage{etex}

% eurosym : symbole euro

\usepackage{eurosym}

\usepackage{tocvsec2}

\setlength\parindent{0pt}

\linespread{1.1}


%%%% debut macro %%%%
\makeatletter
\def\hlinewd#1{%
\noalign{\ifnum0=`}\fi\hrule \@height #1 %
\futurelet\reserved@a\@xhline}
\makeatother
%%%% fin macro %%%%

\newcounter{partie}
\newcounter{sous-partie}

\renewcommand{\thesection}{\Roman{section}}

\newenvironment{partie}[1]
{
\section{#1}
}
{

}


\newcommand{\mymark}[1]{\markboth{\MakeUppercase{#1}}{\MakeUppercase{#1}}}

% Environnements représentants les parties du document

\newenvironment{intro}
{
\section*{Introduction}
\addcontentsline{toc}{section}{Introduction}
\mymark{Introduction}
}
{

}

\newenvironment{remerciements}
{
\section*{Remerciements}
\addcontentsline{toc}{section}{Remerciements}
\mymark{Remerciements}
}
{

}

\newenvironment{conclusion}
{
\section*{Conclusion}
\addcontentsline{toc}{section}{Conclusion}
\mymark{Conclusion}
}
{

}

\newenvironment{resume}
{
\section*{Résumé}
\addcontentsline{toc}{section}{Résumé}
\mymark{Résumé}
}
{

}


\newenvironment{sous-partie}[1]
{
\subsection{#1}
}
{

}

\newenvironment{sous-sous-partie}[1]
{
\subsubsection{#1}
}
{

}

\newenvironment{liste}
{
\vspace{0.2cm}
\begin{list}{$\bullet$\hspace{0.3cm}}{\leftmargin=1.4cm}
}
{
\end{list}
\vspace{0.2cm}
}

\newenvironment{oliste}
{
\vspace{0.2cm}
\begin{enumerate}
}
{
\end{enumerate}
\vspace{0.2cm}
}

\newcounter{numannexe}
\newenvironment{annexes}
{
\newpage
\thispagestyle{empty}
\mbox{}
\newpage
\addcontentsline{toc}{section}{Annexes}
\mymark{Annexes}
\vspace*{\fill}
\begin{center}
\Huge{\textbf{Annexes}}
\end{center}
\vspace*{\fill}
\newpage
\setcounter{numannexe}{0}

\renewenvironment{partie}[1]
{
	\refstepcounter{subsection}
	\subsection*{\arabic{subsection}. ##1}
}
{
}

}
{

}

\newenvironment{annexe}[1]
{
\setcounter{subsection}{0}
\refstepcounter{numannexe}
\addcontentsline{toc}{subsection}{Annexe \arabic{numannexe} : #1}
\mymark{Annexe \arabic{numannexe} : #1}
\section*{Annexe \arabic{numannexe} : #1}
}
{
	\FloatBarrier
}

\newenvironment{sous-annexe}[1]
{
\begin{subappendices}
\subsection{#1}
}
{
\end{subappendices}
}


% Divers

% Permet de faire des cadres dans le document
\newcommand{\cadre}[1]{\renewcommand{\arraystretch}{0.4}\begin{array}{!{\vline}c!{\vline}}\hline\\#1\\\\\hline\end{array}}
\newcommand{\cadret}[1]{\renewcommand{\arraystretch}{0.4}\begin{tabular}{!{\vline}c!{\vline}}\hline\\#1\\\\\hline\end{tabular}}
\newcommand{\semicadre}[1]{\renewcommand{\arraystretch}{0.3}\begin{array}{c!{\vline}}#1\\\\\hline\end{array}}

% Commandes mathématiques

\newcommand{\equivalent}[2]{\!\!\renewcommand{\arraystretch}{1}\begin{array}[t]{c}\Huge\sim\\^{#1\rightarrow#2}\end{array}\!\!}
\newcommand{\tend}[2]{\!\renewcommand{\arraystretch}{1}\begin{array}[t]{c}-\!-\!\!\!\longrightarrow\\^{#1\rightarrow#2} \\ [-1.5ex]\end{array}\!}
\newcommand{\egal}[2]{\!\!\renewcommand{\arraystretch}{1}\begin{array}[t]{c}=\\^{#1\rightarrow#2}\end{array}\!\!}
\newcommand{\fin}{\vspace{0.2cm}\\}
\newcommand{\finq}{\vspace{0.5cm}\\}
\newcommand{\sh}{\mathrm{sh}\,}
\newcommand{\abs}[1]{\left\vert#1\right\vert }
\newcommand{\ch}{\mathrm{ch}\,}

\renewcommand{\o}[1]{\mathrm{o}\!\left(#1\right)}
\newcommand{\etoile}{\hspace*{1cm}$\star$\hspace*{0.5cm}}
\renewcommand{\th}{\mathrm{th}\,}
\renewcommand{\arcsin}{\mathrm{Arcsin}\,}
\renewcommand{\arccos}{\mathrm{Arccos}\,}
\newcommand{\argsh}{\mathrm{Argsh}\,}
\newcommand{\argch}{\mathrm{Argch}\,}
\newcommand{\argth}{\mathrm{Argth}\,}
\newcommand{\rg}{\mathrm{rg}\,}
\renewcommand{\arctan}{\mathrm{Arctan}\,}
\newcommand{\sq}{\hspace*{1.4cm}\stepcounter{sq}(\alph{sq})\hspace*{0.5cm}}
\newcommand{\rcl}{\begin{array}{rcl}}
\newcommand{\ea}{\end{array}}
\newcommand{\str}[1]{\renewcommand{\arraystretch}{#1}}
\renewcommand{\tfrac}[2]{\textstyle\frac{#1}{#2}}
\newcommand{\Cl}[1]{$C^{#1}$}
\newcommand{\mathCl}[1]{C^{#1}}
\renewcommand{\t}[1]{\tilde{#1}}
\renewcommand{\l}{\lambda}
\newcommand{\ds}{\displaystyle}
\newcommand{\R}{\mathbb{R}}
\newcommand{\C}{\mathbb{C}}
\newcommand{\Q}{\mathbb{Q}}
\newcommand{\Z}{\mathbb{Z}}
\newcommand{\N}{\mathbb{N}}
\newcommand{\Ker}{\mathrm{Ker}\,}
\newcommand{\Vect}{\mathrm{Vect}}
\renewcommand{\lvert}{\left\vert}
\renewcommand{\Im}{\mathrm{Im}\,}
\renewcommand{\rvert}{\right\vert}
\newcommand{\mnk}{\mathcal{M}_n(K)}
\newcommand{\mnc}{\mathcal{M}_n(C)}
\newcommand{\ppcm}{\mathrm{ppcm}}
\newcommand{\Tr}{\mathrm{Tr}\,}
\renewcommand{\t}[1]{^t\!#1}
\newcommand{\scal}[2]{\left\langle #1|#2\right\rangle}
\newcommand{\scalindice}[4]{\phantom{\langle}_{#3}\!\left\langle #1|#2\right\rangle_{#4}}
\newcommand{\p}[1]{\left( #1 \right)}
\newcommand{\crochet}[1]{\left[ #1 \right]}
\newcommand{\nr}[1]{\left\|\,#1\,\right\|}
\newcommand{\tab}{\hspace*{1cm}}

\newcommand{\esp}[1]{\mathbb{E}\!\crochet{#1}}
\newcommand{\espcond}[2]{\mathbb{E}_{#1}\!\crochet{#2}}

\renewcommand{\P}[1]{\mathbb{P}\!\p{#1}}
\newcommand{\Pcond}[2]{\mathbb{P}_{#1}\!\p{#2}}

\renewcommand{\binom}[2]{\left(\begin{array}{c}#1\\#2\end{array}\right)}

\newcommand{\car}[1]{\mathbf{1}_{#1}}

\newcommand{\matdd}[4]{\left({\begin{array}{cc} #1 & #2\\ #3 & #4 \ea}\right)\vspace{0.05cm}}

\newcommand{\supp}[1]{\text{supp}\left(#1\right)}

\renewcommand{\ge}{\geqslant}
\renewcommand{\le}{\leqslant}

\newcommand{\lebesgue}{\mathcal{L}}

\newcommand{\Drond}{\mathcal{D}}

\renewcommand{\Re}{\mathcal{R}e}
\newcommand{\obs}[1]{\hat{#1}}
\newcommand{\ket}[1]{\vert #1 \rangle}
\newcommand{\bra}[1]{\langle #1 \vert}
\newcommand{\braindice}[2]{\!\phantom{\langle}_{#2}\!\left\langle #1\right\vert}

\newcommand{\prodscal}[2]{\left\langle#1,#2\right\rangle}

\newcommand{\vect}[2]{\left(\str{1}\begin{array}{cc}#1 & #2 \ea\right)}
\newcommand{\vecttrans}[2]{\left(\str{1}\begin{array}{c}#1 \\ #2 \ea\right)}

\renewcommand{\v}[1]{\underline{#1}}

\newcommand{\Dp}[2]{\dfrac{\partial #1}{\partial #2}}
\newcommand{\grad}{\mathrm{grad}\,}
\newcommand{\vgrad}{\v{\mathrm{grad}}\,}




\def\br{\hfill\break\noindent}
\def\root{ \sqrt{ 1 - v^2/c^2}}
\def\p{\varphi}
\def\t{\theta}
\def\g{\gamma}
\def\a{\alpha}
\def\b{\beta}
\def\d{\delta}


\def\be{\begin{equation}}
\def\ee{\end{equation}}
\def\ba{\begin{eqnarray}}
\def\ea{\end{eqnarray}}
\def\ve{\overrightarrow}
\def\Vi{\ve{V_{\infty}}}
\def\Va{\ve{V_a}}
\def\v{\ve{v}}
\def\P{\ve{P}}
\def\Fa{\ve{F_a}}
\def\ex{\ve{e_x}}
\def\ey{\ve{e_y}}
\def\ez{\ve{e_z}}
\def\er{\ve{e_r}}
\def\et{\ve{e_{\t}}}
\def\ep{\ve{e_{\varphi}}}
\def\tp{\dot{\t}}
\def\tpp{\ddot{\t}}
\def\ro{\rho_0}
\def\M{\ve{M}}
\def\u{\ve{u}}
\def\n{\ve{n}}
\def\up{\ve{u_{\sslash}}}
\def\udeux{\ve{u_2}}
\def\l{\vec{l}}
\def\Da{\ve{D_a}}



\newcommand{\tablefigures}{

%\section*{Introduction}
%\addcontentsline{toc}{section}{Introduction}
%\mymark{Introduction}

\listoffigures

}


\pagestyle{fancy}

\setlength{\headheight}{2cm}

\renewcommand{\headrulewidth}{0.4pt}
\renewcommand{\footrulewidth}{0pt}

\fancyhead{}
\fancyfoot{}

\fancyhead[R]{\includegraphics[height=1.5cm]{./images/Charte_graphique_polytechnique/logo_x_simple.png}}

\fancyfoot[RO,LE]{\thepage}
\fancyhead[L]{\leftmark}


\begin{document}

École polytechnique\hfill Mai 2012\\
Promotion 2010
\thispagestyle{empty}
\newlength{\margeimage}
\setlength{\margeimage}{1cm}
\vspace{\margeimage}
\begin{center}
\includegraphics[width=8cm]{./images/blague.png}
\end{center}
\vspace{\margeimage}
\begin{center}
\begin{Large}
\bf
Projet de programmation INF431\vspace{0.3cm}\\
Rapport de fin de projet
\end{Large}
\end{center}
\vspace*{\fill}
\begin{flushright}
Thomas Brichard\\
Quentin Fiard
\end{flushright}

\clearpage


\settocdepth{subsection}

\tableofcontents

\vspace{1.5cm}

\begin{intro}
\begin{paragraph}{}
A l'ère du numérique et du développement massif d'internet, le plagiat de documents est devenu de plus en plus facile, et donc de plus en plus courant. Le type de documents plagiés est très variable, et peut aller d'un texte au code source d'un logiciel protégé, en passant par le code d'un site web.
\end{paragraph}
\begin{paragraph}{}
Des algorithmes ont été développés pour permettre de lutter contre ce fléau, notamment dans un cadre scolaire ou académique, et sont la plupart du temps fondés sur une comparaison d'une multitude de courts extraits des documents. Ce projet de programmation vise à mettre en pratique un de ces algorithmes, étudié dans un article de Schleimer, Wilkerson et Aiken\footnote{Saul Schleimer, Daniel S. Wilkerson, and Alex Aiken. Winnowing : local algorithme for document fingerprinting. In SIGMOD’03 : \textit{Proceedings of the 2003 ACM SIGMOD International Conference on Management of Data}, pages 76–85, 2003.}.
\end{paragraph}

\begin{paragraph}{}

\end{paragraph}

\end{intro}

\clearpage

\begin{partie}{Organisation du travail de groupe}

Ce projet a été pour nous l'occasion de développer nos capacités de travail en équipe, et d'appréhender le processus de développement d'un programme informatique complexe. Nous détaillons dans cette partie la démarche que nous avons mise en place pour mener ce projet, et les outils qui nous ont permis un développement collaboratif du projet.

\begin{sous-partie}{Démarche mise en place}
\begin{paragraph}{}
Comme indiqué dans l'énoncé, nous avons cherché à décomposer au maximum le développement du projet en étapes élémentaires, à même d'être développées en parallèle.
\end{paragraph}
\begin{paragraph}{}
Nous avons dans un premier temps défini de façon précise l'architecture que nous allions mettre en place pour le projet (présentée dans la partie \ref{architecture}). Nous avons ensuite écrit les interfaces et les structures de données que nous allions utilisées par la suite (nous avons essayé au maximum d'écrire chaque partie indépendante du projet dans un package distinct, par soucis d'encapsulation). Nous avons ensuite développé chacun de notre côté différentes fonctionnalités du programme, avant de tester notre programme sur les exemples proposés.
\end{paragraph}
\end{sous-partie}

\begin{sous-partie}{Système de contrôle de versions}
\begin{figure}[htb]
\centering
\includegraphics[width=4cm]{./images/git.png}
\end{figure}

\begin{paragraph}{}
Pour nous permettre de développer en parallèle les fonctionnalités du programme, nous avons utilisé le système de contrôle de version git. Il nous a permis de fusionner les modifications de chacun au fur et à mesure, et nous à autoriser à tout moment à revenir à une version antérieur du code en cas de bugs de programmation. Il présente également l'avantage de pouvoir être utilisé hors ligne (contrairement à Dropbox par exemple, qui nécessite une connexion internet pour pouvoir conserver les différentes versions des fichiers).
\end{paragraph}

\begin{paragraph}{}
À cause du proxy de l'École, nous avons du mettre en place un répertoire de dépôt sur le réseau local de l'École, auquel git se connecte par ssh à l'aide d'un identifiant élève. La version finale du programme a également été mis en libre accès sur un dépôt Github\footnote{https://github.com/QuentinFiard/Plagiats/}
\end{paragraph}

\end{sous-partie}

\end{partie}

\clearpage

\begin{partie}{Architecture du programme}

\label{architecture}

\end{partie}

\clearpage

\begin{partie}{Présentation des résultats}

\end{partie}

\clearpage

\begin{partie}{Synthèse et critique de l'algorithme}

\end{partie}

\clearpage

\begin{conclusion}

\end{conclusion}


\end{document}
